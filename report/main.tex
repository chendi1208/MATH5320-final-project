% !TEX encoding = UTF-8 Unicode
\documentclass[letterpaper,11pt, oneside]{layout}

%%%%%%%%%%%%%%%%%%%%%%%%%%%%%%%%%%%%%%%%%%%%%%%%
%%%%%%%%%%%%%%%%%%%% Title of Titlepage %%%%%%%%%%%%%%%%%%%%
%%%%%%%%%%%%%%%%%%%%%%%%%%%%%%%%%%%%%%%%%%%%%%%%

\title{Risk Calculation System}
\subtitle{MATH 5320 Financial Risk Management and Regulation Final Project}
\date{\today}

%%%%%%%%%%%%%%%%%%%%%%%%%%%%%%%%%%%%%%%%%%%%%%
%%%%%%%%%%%%%%%%%%%% Document %%%%%%%%%%%%%%%%%%%%
%%%%%%%%%%%%%%%%%%%%%%%%%%%%%%%%%%%%%%%%%%%%%%

\begin{document}
\cleardoublepage
\maketitle

\frontmatter
\begin{spacing}{1}
%\include{body/abstract}

\tableofcontents
\listoffigures
\listoftables
\end{spacing}

%%%%%%%%%%%%%%%%%%%%%%%%%%%%%%%%%%%%%%%%%%%%%%

\mainmatter

\chapter{Executive Summary}
\label{chap:summ}

%%%%%%%%%%%%%%%%%%%%%%%%%%
\section{Purpose of review}
\label{sec:summ:pr}

This is a review of our group’s project of development of a risk calculation system for a user-defined portfolio. This system is used to calculate historical VaR and ES, parametric VaR and ES, as well as Monte Carlo VaR and ES.

%%%%%%%%%%%%%%%%%%%%%%%%%%
\section{Model description}
\label{sec:summ:md}

%%%%%%%%%%%%%%%%%%%%%%%%%%
\section{Current/intended model usage}
\label{sec:summ:mu}

%%%%%%%%%%%%%%%%%%%%%%%%%%
\section{Validation methodology and scope}
\label{sec:summ:vms}

%%%%%%%%%%%%%%%%%%%%%%%%%%
\section{Critical analysis}
\label{sec:summ:ca}

Here are the pros and cons of the models we used respectively:

For Historical VaR and ES: 
\begin{itemize}
\item Pros: Conceptually easy, use the actual returns, can give the operational and analyzable number, etc
\item Cons: Assume the history would replicate the future, might not give a good prediction when the time window is too short, hard to choose the optimal time window, can be affected significantly by shock events or stress scenarios, cannot answer a question of what-if, etc.
\end{itemize}

For Parametric VaR and ES: 
\begin{itemize}
\item Pros: Only need to do the calculation of mu and sigma, the data for the input is easy to obtain, etc.
\item Cons: the assumption of normality might be fatal, still cannot answer the question of what-if.
\end{itemize}

For Monte Carlo VaR and ES: 
\begin{itemize}
\item Pros: Allow for an infinite number of possible scenarios, can answer a question of what-if.
\item Cons: a way more complex analytical tool, model complexity increase in scale, still need a assumptive distribution. 
\end{itemize}

%%%%%%%%%%%%%%%%%%%%%%%%%%%%%%%%%%%%%%%%%%%%%%

\begingroup
\renewcommand{\clearpage}{}
\chapter{Introduction}
\label{chap:intro}
\endgroup
This is a review of our group’s project of development of a risk calculation system for a portfolio comprised of stocks and options. In this system we realized the computation and visualization of historical VaR, GBM VaR, as well as VaR using Monte Carlo Simulation. 

The data we obtained for input is the historical Close Price daily for the stocks that user wants to form a portfolio. And our system do a calculation of VaR using historical simulation, parametric method, as well as Monte Carlo simulation.

%%%%%%%%%%%%%%%%%%%%%%%%%%%%%%%%%%%%%%%%%%%%%%

\begingroup
\renewcommand{\clearpage}{}
\chapter{Product Description}
\label{chap:pd}
\endgroup

%%%%%%%%%%%%%%%%%%%%%%%%%%%%%%%%%%%%%%%%%%%%%%

\begingroup
\renewcommand{\clearpage}{}
\chapter{Model Description}
\label{chap:md}
\endgroup

%%%%%%%%%%%%%%%%%%%%%%%%%%
\section{Modeling theory/assumptions}
\label{sec:md:mt}

%%%%%%%%%%%%%%%%%%%%%%%%%%
\section{Mathematical description}
\label{sec:md:md}

For Historical Simulation model: 

The model is quite intuitive. We assume the future would replicate the past. So if we want to calculate a p level VaR for a predefined window size and horizon, we could calculate the relative return or absolute return of the portfolio and take the $(1-p)^{\text{th}}$ quantile of the return, scale it in accordance to our original investment $S_0$, and take its inverse as our VaR. And take the mean of the $(1-p)^{\text{th}}$ values as ES.

For Parametric model:

We assume the stocks’ price and the portfolio price resemble geometry Brownian motion. And we can calculate the VaR and ES using the following formulas:

\begin{equation}
dS=\mu Sdt+\sigma SdW
\end{equation}
\begin{equation}
E[S_T]=S_0 e^{\mu T}
\end{equation}
\begin{equation}
Var[S_T]=S_0^2( e^{\sigma^2 T}-1)e^{2\mu T}
\end{equation}
\begin{equation}
VaR(S, T, p)=S_0-S_0 e^{\sigma\sqrt{T}\Phi^{-1}(1-p)+ (\mu-\frac{\sigma^2}{2})T}
\end{equation}
\begin{equation}
ES(S, T, p)=S_0(1-e^{\mu T}/(1-p)\times\Phi(\Phi^{-1}(1-p)-\sqrt{T}\sigma))
\end{equation}

For Monte Carlo method:

Given the drift and volatility of the portfolio on a certain day, we can simulate the potential movement of the portfolio using Monte Carlo Simulation 

\begin{equation}
S_T=S_0 e^{(\mu-\frac{\sigma^2}{2})\times \text{horizon days}/252+\sigma W}
\end{equation}

for say, 10000 times (can be defined by user) and then we can obtain VaR and ES by just sorting the potential losses and take the intended quantile.

%%%%%%%%%%%%%%%%%%%%%%%%%%
\section{Calibration methodology}
\label{sec:md:cm}

To estimate the drift and volatility, for a user defined window size, we need to calibrate the historical data as follows:

%%%%%%%%%%%%%%%%%%%%%%%%%%
\section{Model usage}
\label{sec:md:mu}

%%%%%%%%%%%%%%%%%%%%%%%%%%
\section{Model exposure}
\label{sec:md:me}

%%%%%%%%%%%%%%%%%%%%%%%%%%%%%%%%%%%%%%%%%%%%%%

\begingroup
\renewcommand{\clearpage}{}
\chapter{Validation Methodology and Scope}
\label{chap:vms}
\endgroup

%%%%%%%%%%%%%%%%%%%%%%%%%%%%%%%%%%%%%%%%%%%%%%

\begingroup
\renewcommand{\clearpage}{}
\chapter{Validation Results}
\label{chap:vr}
\endgroup

%%%%%%%%%%%%%%%%%%%%%%%%%%%%%%%%%%%%%%%%%%%%%%

\begingroup
\renewcommand{\clearpage}{}
\chapter{Conclusions and Recommendations}
\label{chap:cr}
\endgroup


\appendix
\renewcommand\theequation{\Alph{chapter}--\arabic{equation}}
\renewcommand\thefigure{\Alph{chapter}--\arabic{figure}}
\renewcommand\thetable{\Alph{chapter}--\arabic{table}}

%\include{readme}
%\include{code}

%%%%%%%%%%%%%%%%%%%%%%%%%%%%%%%%%%%%%%%%%%%%%%

\backmatter

%\bibliography{ref}

%%%%%%%%%%%%%%%%%%%%%%%%%%%%%%%%%%%%%%%%%%%%%%

\end{document}
